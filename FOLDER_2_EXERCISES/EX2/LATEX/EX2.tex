%%%%%%%%%%%%%%%%%%
% DOCUMENT CLASS %
%%%%%%%%%%%%%%%%%%
\documentclass{article}

%%%%%%%%%%%%
% PACKAGES %
%%%%%%%%%%%%
\usepackage{hyperref}

%%%%%%%%%%%%%%%%%%
% BEGIN DOCUMENT %
%%%%%%%%%%%%%%%%%%
\begin{document}

%%%%%%%%%
% TITLE %
%%%%%%%%%
\title{Exercise 2}

%%%%%%%%%%%%%%%%%%%%%%%%%%%%%
% AUTHOR = COURSE NAME HERE %
%%%%%%%%%%%%%%%%%%%%%%%%%%%%%
\author{Compilation 1450:3101}

%%%%%%%%%%%%%%%%%%%%%%%%%%%%%%%
% DATE = SUBMISSION DATE HERE %
%%%%%%%%%%%%%%%%%%%%%%%%%%%%%%%
\date{Due 11/4/2018}

%%%%%%%%%
% TITLE %
%%%%%%%%%
\maketitle

%%%%%%%%%%%%%%%%%%%%%%%%%%%
% SECTION :: Introduction %
%%%%%%%%%%%%%%%%%%%%%%%%%%%
\section{Introduction}
We continue our journey of building a compiler
for the invented object oriented language RioMare.
Remember that the entire specification of RioMare appears
inside the relevant folder of the course website.
In order to make this document self contained,
all the information needed to complete the second exercise is brought here again.
%%%%%%%%%%%%%%%%%%%%%%%%%%%%%%%%%%%%%
% SECTION :: Programming Assignment %
%%%%%%%%%%%%%%%%%%%%%%%%%%%%%%%%%%%%%
\section{Programming Assignment}
The second exercise implements a \href{http://www2.cs.tum.edu/projects/cup/}{CUP} based
parser on top of your \href{http://jflex.de/}{JFlex} scanner from the exercise $1$.
The input for the parser is a single text file containing a RioMare program,
and the output is a (single) text file indicating whether the input program
is syntactically valid or not. In addition to that,
whenever the input program has correct syntax,
the parser should internally create the abstract syntax tree (AST).
Currently, the course repository contains a simple skeleton
parser, that indicates whether the input program has correct syntax,
and internally builds an AST for a small subset of RioMare.
As always, you are encouraged to work your way up from there,
but feel free to write the whole exercise from scratch if you want to.
Note also, that the AST will not be checked in exercise $2$.
It is needed for later phases (semantic analyzer and code generation)
but the best time to design and implement the AST is exercise $2$.
%%%%%%%%%%%%%%%%%%%%%%%%%%%%%%%%%
% SECTION :: The RioMare Syntax %
%%%%%%%%%%%%%%%%%%%%%%%%%%%%%%%%%
\section{The RioMare Syntax}
Table \ref{Table_CFG_Of_RioMare} specifies the context free grammar of RioMare.
You will need to feed this grammar to \href{http://www2.cs.tum.edu/projects/cup/}{CUP},
and make sure there are no shift-reduce conflicts.
The operator precedence is listed in Table
\ref{Table_Binary_Operators_Of_RioMare}. 
\begin{table}[h]
\centering
\begin{tabular}{ l c l }
%%%%%%%%%%%%%%%%%%%%%%%%%%%%%%%
Program  & $::=$ & dec$^{+}$ \\
%%%%%%%%%%%%%%%%%%%%%%%%%%%%%%%
\\
%%%%%%%%%%%%%%%%%%%%%%%%%%%%%%%%%%%%%%%%%%%%%%%%%%%%%
dec      & $::=$ & funcDec $|$ varDec $|$ classDec \\
%%%%%%%%%%%%%%%%%%%%%%%%%%%%%%%%%%%%%%%%%%%%%%%%%%%%%
\\
%%%%%%%%%%%%%%%%%%%%%%%%%%%%%%%%%%%%%%%%%%%%%%%%%%
varDec   & $::=$ & ID ID $[$ ASSIGN exp $]$ ';' \\
%%%%%%%%%%%%%%%%%%%%%%%%%%%%%%%%%%%%%%%%%%%%%%%%%%
%%%%%%%%%%%%%%%%%%%%%%%%%%%%%%%%%%%%%%%%%%%%%%%%%%%%%%%%%%%%%%%%%%%%%%%%%%%
funcDec  & $::=$ & ID ID $'('$ $[$ ID ID $[$ ',' ID ID $]^{*}$ $]$ $')'$ %%
                   $'\{'$ stmt   $[$ stmt $]^{*}$ $'\}'$                 \\
%%%%%%%%%%%%%%%%%%%%%%%%%%%%%%%%%%%%%%%%%%%%%%%%%%%%%%%%%%%%%%%%%%%%%%%%%%%
%%%%%%%%%%%%%%%%%%%%%%%%%%%%%%%%%%%%%%%%%%%%%%%%%%%%%%%%%%%%%%%%%%%%%%%%%%%%%%%%%%%%%%%%%
classDec & $::=$ & CLASS ID $[$ EXTENDS ID $]$ $'\{'$ cField $[$ cField $]^{*}$ $'\}'$ \\
%%%%%%%%%%%%%%%%%%%%%%%%%%%%%%%%%%%%%%%%%%%%%%%%%%%%%%%%%%%%%%%%%%%%%%%%%%%%%%%%%%%%%%%%%
\\
%%%%%%%%%%%%%%%%%%%%%%%%%%%%%%%%%%%%%%%%%%%%%%%%%%%%%%%%%%%%%%%%%%%%%%%%%%%%%%%%%%%%
exp      & $::=$ & var                                                            \\
         & $::=$ & $'('$ exp $')'$                                                \\
         & $::=$ & exp BINOP exp                                                  \\
         & $::=$ & $[$ var '.' $]$ ID $'('$ $[$ exp $[$ ',' exp $]^{*}$ $]$ $')'$ \\
         & $::=$ & $['-']$ INT $|$ NIL                                            \\
%%%%%%%%%%%%%%%%%%%%%%%%%%%%%%%%%%%%%%%%%%%%%%%%%%%%%%%%%%%%%%%%%%%%%%%%%%%%%%%%%%%%%
\\
%%%%%%%%%%%%%%%%%%%%%%%%%%%%%%%%%%%%%%%%%
var      & $::=$ & ID                  \\
         & $::=$ & var '.' ID          \\
%%%%%%%%%%%%%%%%%%%%%%%%%%%%%%%%%%%%%%%%%
\\  
%%%%%%%%%%%%%%%%%%%%%%%%%%%%%%%%%%%%%%%%%%%%%%%%%%%%%%%%%%%%%%%%%%%%%%%%%%%%%%%%%%%%%%%%
stmt     & $::=$ & varDec                                                             \\
         & $::=$ & var ASSIGN exp ';'                                                 \\
         & $::=$ & var ASSIGN NEW ID ';'                                              \\
         & $::=$ & RETURN $[$ exp $]$ ';'                                             \\
         & $::=$ & IF $'('$ exp $')'$ $'\{'$ stmt $[$ stmt $]^{*}$ $'\}'$             \\
         & $::=$ & WHILE $'('$ exp $')'$ $'\{'$ stmt $[$ stmt $]^{*}$ $'\}'$          \\
         & $::=$ & $[$ var '.' $]$ ID $'('$ $[$ exp $[$ ',' exp $]^{*}$ $]$ $')'$ ';' \\
%%%%%%%%%%%%%%%%%%%%%%%%%%%%%%%%%%%%%%%%%%%%%%%%%%%%%%%%%%%%%%%%%%%%%%%%%%%%%%%%%%%%%%%%
\\
%%%%%%%%%%%%%%%%%%%%%%%%%%%%%%%%%%%%%%%%
cField   & $::=$ & varDec $|$ funcDec \\
%%%%%%%%%%%%%%%%%%%%%%%%%%%%%%%%%%%%%%%%
%%%%%%%%%%%%%%%%%%%%%%%%%%%%%%%%%%%%%%%%%%%%%%%%%%%%%%%%%%%%%%%%%%%%%%%%%
BINOP    & $::=$ & $+$ $|$ $-$ $|$ $*$ $|$ $/$ $|$ $<$ $|$ $>$ $|$ $=$ \\
INT      & $::=$ & $[1-9][0-9]^{*}$ $|$ $0$                            \\
%%%%%%%%%%%%%%%%%%%%%%%%%%%%%%%%%%%%%%%%%%%%%%%%%%%%%%%%%%%%%%%%%%%%%%%%%
\\
\end{tabular}
\caption{
Context free grammar for the RioMare programming language.
\label{Table_CFG_Of_RioMare}}
\end{table}
\begin{table}[h]
\centering
\begin{tabular}{ |c|c|l|l| }
\hline
Precedence & Operator & Description & Associativity \\
\hline
\hline
%%%%%%%%%%%%%%%%%%%%%%%%%%%%%%%%%%%%%%%%%%%%%%%%%%%%%%%%
1          & $:=$            & assign         &       \\
%%%%%%%%%%%%%%%%%%%%%%%%%%%%%%%%%%%%%%%%%%%%%%%%%%%%%%%%
\hline
%%%%%%%%%%%%%%%%%%%%%%%%%%%%%%%%%%%%%%%%%%%%%%%%%%%%%%%%
2          & $=$             & equals         & left  \\
%%%%%%%%%%%%%%%%%%%%%%%%%%%%%%%%%%%%%%%%%%%%%%%%%%%%%%%%
\hline
%%%%%%%%%%%%%%%%%%%%%%%%%%%%%%%%%%%%%%%%%%%%%%%%%%%%%%%%
3          & $<,>$           &                & left  \\
%%%%%%%%%%%%%%%%%%%%%%%%%%%%%%%%%%%%%%%%%%%%%%%%%%%%%%%%
\hline
%%%%%%%%%%%%%%%%%%%%%%%%%%%%%%%%%%%%%%%%%%%%%%%%%%%%%%%%
4          & $+,-$           &                & left  \\
%%%%%%%%%%%%%%%%%%%%%%%%%%%%%%%%%%%%%%%%%%%%%%%%%%%%%%%%
\hline
%%%%%%%%%%%%%%%%%%%%%%%%%%%%%%%%%%%%%%%%%%%%%%%%%%%%%%%%
5          & $*,/$           &                & left  \\
%%%%%%%%%%%%%%%%%%%%%%%%%%%%%%%%%%%%%%%%%%%%%%%%%%%%%%%%
\hline
%%%%%%%%%%%%%%%%%%%%%%%%%%%%%%%%%%%%%%%%%%%%%%%%%%%%%%%%
6          & $($             & function call  &       \\
%%%%%%%%%%%%%%%%%%%%%%%%%%%%%%%%%%%%%%%%%%%%%%%%%%%%%%%%
\hline
%%%%%%%%%%%%%%%%%%%%%%%%%%%%%%%%%%%%%%%%%%%%%%%%%%%%%%%%
7          & $.$     & field access   & left          \\
%%%%%%%%%%%%%%%%%%%%%%%%%%%%%%%%%%%%%%%%%%%%%%%%%%%%%%%%
\hline
\end{tabular}
\caption{
Binary operators of RioMare along with their associativity and precedence.
$1$ stands for the lowest precedence, and $7$ for the highest.
\label{Table_Binary_Operators_Of_RioMare}}
\end{table}
%%%%%%%%%%%%%%%%%%%%
% SECTION :: Input %
%%%%%%%%%%%%%%%%%%%%
\section{Input}
The input for this exercise is a single text file, the input RioMare program.
%%%%%%%%%%%%%%%%%%%%%
% SECTION :: Output %
%%%%%%%%%%%%%%%%%%%%%
\section{Output}
The output is a \textit{single} text file that contains a \textit{single} word.
Either OK when the input program has correct syntax,
or otherwise ERROR(\textit{location}), where \textit{location}
is the line number of the \textit{first} error that was encountered.
%%%%%%%%%%%%%%%%%%%%%%%%%%%%%%%%%%%%
% SECTION :: Submission Guidelines %
%%%%%%%%%%%%%%%%%%%%%%%%%%%%%%%%%%%%
\section{Submission Guidelines}
The skeleton code for this exercise resides (as usual)
in subdirectory EX2 of the course repository.
You need to add the relevant derivation rules and AST constructors.
COMPILATION/EX2 should contain a makefile building your source files to a
runnable jar file called PARSER (note the lack of the .jar suffix).
Feel free to use the makefile supplied in the course repository,
or write a new one if you want to. 
Before you submit, make sure that your exercise compiles and runs
on \href{http://releases.ubuntu.com/14.04/}{Ubuntu 14.04.5 LTS}.
This is the formal running environment of the course.
\paragraph{Execution parameters}
PARSER receives $2$ input file names:\\ \\
InputRioMareProgram.txt\\
OutputStatus.txt

\end{document}
