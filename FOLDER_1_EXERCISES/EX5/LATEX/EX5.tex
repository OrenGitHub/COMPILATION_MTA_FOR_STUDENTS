\documentclass{article}
\usepackage{hyperref}
\title{Exercise 4}
\author{IR and Code Generation}
\date{Due: January 17}
\begin{document}
\maketitle
\section*{Programming Assignment}
Fill in the missing parts to obtain a full functioning IR and code generation.
For the IR part, you need to handle function declarations,
and implement a run time protection mechanism against the following hazardous operations:
$(1)$ dereferencing a field of an unallocated object.
$(2)$ accessing an array outside of its bounds.
When the program identifies one of these behaviors, it should print a relevant error message and abort immediately.
For the code generation part you will need to implement two things:
$(1)$ allocation of arrays, records and classes.
$(2)$ calling simple functions and class member functions

\section*{Details}
Implement the following functions:\\
$(1)$ \verb"IR_transFuncDec" \\
$(2)$ \verb"IR_transVarExp" \\
$(3)$ \verb"MIPS_ASM_ALLOCATE_RECORD_IMPLEMENTATION"\\
$(4)$ \verb"MIPS_ASM_ALLOCATE_ARRAY_IMPLEMENTATION"\\
$(5)$ \verb"MIPS_ASM_CodeGeneration_Call"\\
Don't forget to support class member functions call as well.

\section*{Submission Guidelines}
Open an account on \href{https://github.com/}{GitHub}.
Then, visit the
\href{https://education.github.com/discount_requests/new}{academic discount page}
to enable the free creation of private repositories.
One team member should create a new private repository called COMPILATION,
and then invite other team members and me as collaborators.
My username is OrenGitHub.
Please make sure the uppermost folder (COMPILATION) contains a text file ``IDS.text" with the IDs of all team members (one ID per line).
Inside COMPILATION create a sub directory EX4 which will contain your source code and makefile.
Your program should be called compiler, and accept two input file names:\\ \\
\verb"compiler" \verb"<Input.StarKist.txt>" \verb"<Output.PseudoMIPS.txt>"
\end{document}
